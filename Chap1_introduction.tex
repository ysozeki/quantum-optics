\chapter{Introduction}
Quantum optics deals with quantum nature of light, where light is regarded as an ensemble of particles called photons. The quantum nature of light appears as `noise' in various applications in optics and photonics such as optical measurement, optical manipulation, and optical communications, leading to the physical limit called quantum limit on the performance or precision achieved by these methods. To push the performance of various methods to the physical limit, it is crucial to understand the quantum limit. Furthermore, quantum nature of light is extensively utilized to develop various quantum technologies such as quantum cryptography, quantum teleportation, and quantum computing. 

Optical measurement always involves the detection of light. It is categorized into direct detection and homodyne/heterodyne detection. Direct detection gives the intensity of light, and homodyne/heterodyne detection which gives the amplitude of light. Furthermore, optical amplification is often utilized before photodetection to mitigate the effect of detector noise. In every case, the signal-to-noise ratio is ultimately limited by quantum noise, and becomes the same order as the number of photons. This limit cannot be surpassed by classical (i.e., non-quantum) methods, while various methods to surpass the limit is developed by using quantum optics.

This lecture note aims at dealing with quantum noise of light. Chapter 2 summarizes the noise in optical measurements. Chapter 3 introduces quantum harmonic oscillators, which is an analogue of light in quantum optics. Chapter 4 describes the evolution of quantum states. Chapter 5 explains the quantization of light. Chapter 6 introduces representative quantum states. Chapter 7 describes two main interactions in quantum optics: mode mixing and parametric amplifications. Chapter 8 explains quantum optical treatment of optical measurements. Appendix contains basic quantum mechanics, Wigner function, and so on. Variables and operators are summarized in Table \ref{table:variables}.



This lecture note was first prepared in Japanese in 2017, in which I referred to Prof. Kazuro Kikuchi's lecture note. Then I rearranged the contents in English in 2020. In particular, I tried to provide an intuitive picture of quantum optics by extensively using wavefunctions. I often hear that quantum optics is abstract; we can somehow calculate various properties of light using bra-ket and operators, while the physics behind them are quite unclear. Instead, typical quantum-optical calculation using operators can be understood as the rotation and distortion of wavefunctions. I hope that this lecture memo provide such intuitive pictures along with the calculation procedures.  I appreciate many comments and feedback from students in my research group and in the lecture. Any feedback is appreciated at \texttt{https://github.com/ysozeki/quantum-optics}.

\begin{table}
\caption{List of variables and operators}	
\begin{center}
\begin{tabular}{c l}
\hline
	Variable or operator & Explanation \\
	\hline \hline
	$\hat x$ & Normalized position (real part of complex amplitude)\\
	$\hat p$ & Normalized momentum (imaginary part of complex amplitude)\\
	$\hat a$ & Normalized complex amplitude\\
	$\hat a ^\dagger \hat a = \hat n$ & Number of photons\\
	$q$ & Elementary charge\\
	$\hbar$ & Planck constant divided by $2\pi$\\
	$P$ & Optical power (energy per unit time)\\
	$I$ & Current (charge per unit time)\\
	$\eta$ & Quantum efficiency\\
	$\tau$ & Time duration of a single time-frequency mode\\
	$B$ & Nyquist frequency\\
	$T$ & Temperature\\
\hline
\end{tabular}
\label{table:variables}
\end{center}
\end{table}

