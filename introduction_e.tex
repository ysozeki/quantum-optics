\chapter{Introduction}

Light is regarded as an ensemble of particles called photons. The particle nature of light appears as 'noise' in various applications of light waves such as optical measurement, optical manipulation, and optical communications, leading to the physical limit of the performance or precision achieved by these methods. To enhance 



%光は光子(フォトン)と呼ばれる粒子の集まりである。光が粒子であるということ(量子性)は、光を用いて通信、制御、計測を行う際において雑音として現れ、性能や精度の物理限界を与える。光を用いた計測技術の高精度化・高感度化や光通信の大容量化を図るうえで、その限界を知り、限界を高めていくことは重要であり、そのためには光の量子性に関する理解も欠かせない。また、近年では光の量子性を積極的に活用することで、量子暗号、量子テレポーテーション、量子コンピューティング等の技術の開発も進められている。これらの光の量子性を扱う学問を量子光学という。

%光の検出法には様々な種類があり、それは光の強度を検出する光子数検出法と、光の干渉を用いて光の電界振幅を検出するホモダイン・ヘテロダイン法に大別できる。また、光検出器そのものの雑音の影響を抑制するために、光を検出する前に光増幅を行う場合もある。いずれの場合も、検出に伴う雑音は究極的には量子雑音で制限される。レーザ光など、古典的な波動としての光を用いる場合、光子数を$n$とすると、その信号対雑音比は$n$のオーダーになる。これは、古典的な光を使う限り、どのような検出器、検出方法を用いたとしても超えることのできない壁である。一方、この壁を超える手法として非古典的な光を活用する研究も進められている。これらを統一的に理解する上で、量子光学に関する知識は不可欠である。

%本講義メモは、光の量子雑音の考え方について理解することを目的とする。まず、量子雑音について簡単におさらいしたのち、光の量子論で基本となる放物線ポテンシャルの量子力学と、その記述方法をまとめる。次に、光の光子数が確定した状態や、古典的な状態であるコヒーレント状態など、代表的な光の量子状態についてその性質を議論する。その後、光の計測とそれに伴う量子雑音について述べる。最後に、光増幅に伴う雑音の発生について議論する。

%なお、本講義は電気系の修士1年生を対象としている。

%本講義メモは、2015年度までの菊池和朗教授の講義メモを参考に、2017年度に準備したものです。これまでに多くの学生さんから多数の誤りや分かりにくい点のコメントをいただきました。深く感謝いたします。今後も、お気づきの点があれば\texttt{ozeki@ee.t.u-tokyo.ac.jp}までコメントをお寄せください。

