\begin{comment}
\chapter{修正記録}
\begin{itemize}
	\item 2017/7/7 書き始め。
	\item 2017/8/17 光増幅まで一通り書いた。
\end{itemize}
8/17 全体の構成についてのレビュー
\begin{itemize}
	\item 干渉検出、ビームスプリッタ等は1章に持ってくるべき。
	\item 量子力学のおさらいが長い。証明はAppendixに持っていく。
	\item 重要なメッセージを章の最後に書く。
\end{itemize}

2章のまとめを書いていない

疑問:ビームスプリッタのハミルトニアンの固有状態と固有値は?

その前に、変位演算子のハミルトニアンの固有状態は?$\alpha = -i$とすると、
\begin{equation}
  \hat H_d = \hbar(\hat a + \hat a^\dagger)
\end{equation}
これはエルミートなので必ず固有状態をもつ。
\begin{equation}
  \ket \psi = \delta(Q)
\end{equation}
固有値は$Q$。なので、変位演算子によって$-P$方向にシフトする。

このことを用いてバランスド検出の説明を試みる?

大事な項目を最後にまとめる。

最終的には、$\hat Q, \hat P$での表現を全て$\hat x, \hat y$に一本化したい。

位置$Q$を基底とする場合

$\hat H_c = \hbar(\hat a \hat b^\dagger + \hat a^\dagger \hat b)$の固有状態:$\ket{\phi_{1\pm}} = \frac 1 {\sqrt 2}(\ket 1 \ket 0 \pm \ket 0\ket 1) $(シングルフォトンの場合。$\ket {\phi_\pm}$は直積で書けない。これはエンタングル状態であることを示している。)

$\ket {\phi_{2\pm}} = \ket 2 \ket 0 \pm \sqrt 2\ket 1 \ket 1 + \ket 0 \ket 2$

$\hat H_c\ket{\phi_{2\pm}} = \hbar(\sqrt 2 \ket 1 \ket 1 \pm 2\ket 0 \ket 2 \pm 2\ket 2 \ket 0 + \sqrt 2 \ket 1 \ket 1) = \pm \sqrt 2\hbar \ket {\phi_{2\pm}}$ 

直積についての説明を書く。

参考文献リストを作っておく。


\textbf{行うことリスト}
\begin{itemize}
	\item 絵を描く。
\end{itemize}


\begin{table}
	
\begin{center}
	
\caption{操作の演算子とその意味}
\begin{tabular}{c c}
\hline
演算子 & 意味\\
\hline \hline
$\hat H = \hbar \omega (\hat a_1^2 + \hat a_2^2) = \hbar \omega (\hat a^\dagger \hat a + 1/2)$& 調和振動子のハミルトニアン\\
$U(t) = \exp(\hat H t/ i\hbar )$& 時間発展演算子\\
$D(\alpha) = \exp(\alpha \hat a^\dagger - \alpha^*\hat a)$& 変位演算子\\
$\hat H_d = i\hbar(\alpha \hat a^\dagger - \alpha^* \hat a)$& 変位のハミルトニアン\\
$\hat H_c = \hbar(\hat a \hat b^\dagger + \hat a^\dagger \hat b)$& 2モード結合のハミルトニアン\\
$\hat H_a = \hbar(\hat a \hat b + \hat a^\dagger \hat b^\dagger)$& 光増幅のハミルトニアン\\
\hline
\end{tabular}

\end{center}
\end{table}

\begin{table}
\begin{center}
	\caption{特殊な量子状態}
	\begin{tabular}{c c}
	\hline
		ケット & 意味\\
		\hline \hline
		$\ket n$ & 光子数$n$の光子数状態\\
		$\ket \alpha$ & 複素振幅$\alpha$のコヒーレント状態\\
		\hline
	\end{tabular}
\end{center}
\end{table}
\end{comment}

\chapter{はじめに}


光は光子(フォトン)と呼ばれる粒子の集まりである。光が粒子であるということ(量子性)は、光を用いて通信、制御、計測を行う際において雑音として現れ、性能や精度の物理限界を与える。光を用いた計測技術の高精度化・高感度化や光通信の大容量化を図るうえで、その限界を知り、限界を高めていくことは重要であり、そのためには光の量子性に関する理解も欠かせない。また、近年では光の量子性を積極的に活用することで、量子暗号、量子テレポーテーション、量子コンピューティング等の技術の開発も進められている。これらの光の量子性を扱う学問を量子光学という。

光の検出法には様々な種類があり、それは光の強度を検出する光子数検出法と、光の干渉を用いて光の電界振幅を検出するホモダイン・ヘテロダイン法に大別できる。また、光検出器そのものの雑音の影響を抑制するために、光を検出する前に光増幅を行う場合もある。いずれの場合も、検出に伴う雑音は究極的には量子雑音で制限される。レーザ光など、古典的な波動としての光を用いる場合、光子数を$n$とすると、その信号対雑音比は$n$のオーダーになる。これは、古典的な光を使う限り、どのような検出器、検出方法を用いたとしても超えることのできない壁である。一方、この壁を超える手法として非古典的な光を活用する研究も進められている。これらを統一的に理解する上で、量子光学に関する知識は不可欠である。

本講義メモは、光の量子雑音の考え方について理解することを目的とする。まず、量子雑音について簡単におさらいしたのち、光の量子論で基本となる放物線ポテンシャルの量子力学と、その記述方法をまとめる。次に、光の光子数が確定した状態や、古典的な状態であるコヒーレント状態など、代表的な光の量子状態についてその性質を議論する。その後、光の計測とそれに伴う量子雑音について述べる。最後に、光増幅に伴う雑音の発生について議論する。

なお、本講義は電気系の修士1年生を対象としている。

本講義メモは、2015年度までの菊池和朗教授の講義メモを参考に、2017年度に準備したものです。これまでに多くの学生さんから多数の誤りや分かりにくい点のコメントをいただきました。深く感謝いたします。今後も、お気づきの点があれば\texttt{ozeki@ee.t.u-tokyo.ac.jp}までコメントをお寄せください。

